%% -*- coding: utf-8 -*-
\documentclass[12pt,a4paper]{scrartcl} 
\usepackage[utf8]{inputenc}
\usepackage[english,russian]{babel}
\usepackage{indentfirst}
\usepackage{misccorr}
\usepackage{graphicx}
\usepackage{amsmath}
\documentclass{article}
\usepackage[T2A]{fontenc}
\usepackage{amsmath}
\usepackage{geometry}

\begin{document}
	\begin{titlepage}
		\begin{center}
			\largeМИНИСТЕРСТВО НАУКИ И ВЫСШЕГО ОБРАЗОВАНИЯ РОССИЙСКОЙ ФЕДЕРАЦИИ
			Федеральное государственное бюджетное образовательное учреждение высшего образования
			
			\textbf{АДЫГЕЙСКИЙ ГОСУДАРСТВЕННЫЙ УНИВЕРСИТЕТ}
			\vspace{0.25cm}
			
			Инженерно-физический факультет
			
			Кафедра автоматизированных систем обработки информации и управления
			\vfill

			\vfill
			
			\textsc{Отчет по практике}\\[5mm]
			
			{\LARGE Программаная реализация численного метода \textit{Найти ранг матрицы.}}
			\bigskip
			
			1 курс, группа 1ИВТ АСОИУ
		\end{center}
		\vfill
		
		\newlength{\ML}
		\settowidth{\ML}{«\underline{\hspace{0.7cm}}» \underline{\hspace{2cm}}}
		\hfill\begin{minipage}{0.5\textwidth}
			Выполнил:\\
			\underline{\hspace{\ML}} В.\,Р.~Хайрулин\\
			«\underline{\hspace{0.7cm}}» \underline{\hspace{2cm}} 2024 г.
		\end{minipage}%
		\bigskip
		
		\hfill\begin{minipage}{0.5\textwidth}
			Руководитель:\\
			\underline{\hspace{\ML}} С.\,В.~Теплоухов\\
			«\underline{\hspace{0.7cm}}» \underline{\hspace{2cm}} 2024 г.
		\end{minipage}%
		\vfill
		
		\begin{center}
			Майкоп, 2024 г.
		\end{center}
	\end{titlepage}

\section*{Введение}
В данной работе реализован алгоритм нахождения ранга матрицы с использованием метода Гаусса. Программа написана на языке C++ и использует динамические массивы для хранения матрицы.

\section*{Описание кода}
Программа состоит из следующих частей:
\begin{enumerate}
    \item Подключение необходимых библиотек.
    \item Функция для вывода матрицы.
    \item Функция для приведения матрицы к ступенчатому виду методом Гаусса.
    \item Основная функция, в которой осуществляется ввод матрицы, вызов функций и вывод результатов.
\end{enumerate}

\subsection*{Подключение библиотек}
Подключаются стандартные библиотеки для ввода-вывода, обработки исключений и манипуляции выводом:
\begin{verbatim}
#include <iostream>
#include <iomanip>
#include <stdexcept>
using namespace std;
\end{verbatim}

\subsection*{Функция для вывода матрицы}
Функция \texttt{printMatrix} выводит матрицу в консоль, красиво форматируя вывод:
\begin{verbatim}
void printMatrix(double** matrix, int rows, int cols) {
    for (int i = 0; i < rows; ++i) {
        for (int j = 0; j < cols; ++j) {
            cout << setw(10) << matrix[i][j] << " ";
        }
        cout << endl;
    }
}
\end{verbatim}

\subsection*{Функция приведения к ступенчатому виду}
Функция \texttt{gaussianElimination} реализует метод Гаусса и возвращает ранг матрицы:
\begin{verbatim}
int gaussianElimination(double** matrix, int rows, int cols) {
    int rank = 0;
    for (int col = 0; col < cols; ++col) {
        int pivot = rank;
        while (pivot < rows && matrix[pivot][col] == 0) {
            ++pivot;
        }
        if (pivot == rows) {
            continue;
        }
        if (pivot != rank) {
            swap(matrix[rank], matrix[pivot]);
        }
        double pivotValue = matrix[rank][col];
        if (pivotValue == 0) {
            throw runtime_error("Pivot element is zero during normalization.");
        }
        for (int j = col; j < cols; ++j) {
            matrix[rank][j] /= pivotValue;
        }
        for (int i = rank + 1; i < rows; ++i) {
            double factor = matrix[i][col];
            for (int j = col; j < cols; ++j) {
                matrix[i][j] -= factor * matrix[rank][j];
            }
        }
        ++rank;
    }
    return rank;
}
\end{verbatim}

\subsection*{Основная функция}
Основная функция организует ввод данных, вызов функций и вывод результатов:
\begin{verbatim}
int main() {
    try {
        int rows, cols;
        cout << "Введите количество строк и столбцов матрицы: ";
        cin >> rows >> cols;
        if (rows <= 0 || cols <= 0) {
            throw invalid_argument("Количество строк и столбцов должно быть положительным.");
        }
        double** matrix = new double*[rows];
        for (int i = 0; i < rows; ++i) {
            matrix[i] = new double[cols];
        }
        cout << "Введите элементы матрицы:\n";
        for (int i = 0; i < rows; ++i) {
            for (int j = 0; j < cols; ++j) {
                if (!(cin >> matrix[i][j])) {
                    throw runtime_error("Ошибка ввода. Пожалуйста, вводите только числовые значения.");
                }
            }
        }
        cout << "Исходная матрица:\n";
        printMatrix(matrix, rows, cols);
        int rank = gaussianElimination(matrix, rows, cols);
        cout << "Матрица после приведения к ступенчатому виду:\n";
        printMatrix(matrix, rows, cols);
        cout << "Ранг матрицы: " << rank << endl;
        for (int i = 0; i < rows; ++i) {
            delete[] matrix[i];
        }
        delete[] matrix;
    } catch (const exception& e) {
        cerr << "Ошибка: " << e.what() << endl;
        return 1;
    }
    return 0;
}
\end{verbatim}

\end{document}
